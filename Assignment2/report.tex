

\documentclass[a4paper]{article}

\usepackage{amsmath}
\usepackage{hyperref}
\usepackage{biblatex}
\usepackage{enumerate}
\usepackage{graphicx}
\usepackage{stmaryrd}
\usepackage[dvipsnames]{xcolor}
\usepackage{listings}
\usepackage{caption}
\usepackage{subcaption}
\usepackage{booktabs}


\addbibresource{refs.bib}

\begin{document}

\author{Ola Bratt \\
  \href{mailto:ola.bratt@gmail.com}{ola.bratt@gmail.com}
  \and
  Patrick Attimont \\
  \href{patrickattimont@gmail.com}{patrickattimont@gmail.com}
}

\title{DAT565/DIT407 Assignment 2}
\date{2024-01-30}

\maketitle

This paper is addressing the assignment 2 study queries within the \emph{Introduction to Data Science \& AI} course, DIT407 at 
the University of Gothenburg and DAT565 at Chalmers. The main source of information for this project
is derived from the lectures and Skiena~\cite{Skiena:2024}. 
\section*{Problem 1: Scrapping house prices}
Problem 1 have been solved using BeautifulSoup together with simple string operations such as 
\begin{verbatim}
  split, replace and strip, 
\end{verbatim}
also regaular expressions have been used to idefity certain information. The code can be found in the appendix.

\section*{Problem 2: Analyzing 2022 house sales}
To caluculate the five-number summary of the closing prices of the houses prices we simply used 
\begin{verbatim}
describe()
\end{verbatim}
on the dataframe containing the closing prices. The result can be seen in Table~\ref{tabular:five_number_summary}.\\

When generating the histogram depicting closing prices (see Figure~\ref{fig:histogram_closing_price}), we employed the "square root method" to determine the bin size. This method was chosen for its ability to unveil trends while maintaining a balance, as larger bins would obscure relevant features, such as the dip around 4.000.000 kr. The resulting plot exhibits a right skew, which is expected given the scarcity of high-priced houses.

Figure~\ref{fig:closing_price_house_ares} displays the relationship between closing prices and house areas, while Figure~\ref{fig:closing_price_house_ares_color} illustrates the same relationship, with the number of rooms colorized.


\begin{table}
  \begin{center}
  \begin{tabular}{c c}
    min & 250000 \\
    \text{25\%} & 3200000 \\
    \text{50\%} & 4100000 \\
    \text{75\%} & 5035000 \\
    max & 21000000 \\
  \end{tabular}
\end{center}
\caption{Five-number summary of closing prices}
  \label{tabular:five_number_summary}
\end{table}

\newpage

\begin{figure}
  \centering
  \begin{subfigure}[a]{\textwidth}
      \centering
      \includegraphics[width=\textwidth]{histogram_closing_price.pdf}
      \caption{Closing prices of houses}
      \label{fig:histogram_closing_price}
  \end{subfigure}
  \vfill
  \begin{subfigure}[b]{\textwidth}
      \centering
      \includegraphics[width=\textwidth]{closing_price_house_ares.pdf}
      \caption{Closing price vs house area}
      \label{fig:closing_price_house_ares}
  \end{subfigure}
  \vfill
  \begin{subfigure}[c]{\textwidth}
      \centering
      \includegraphics[width=\textwidth]{closing_price_house_ares_color.pdf}
      \caption{Closing price vs house area with color}
      \label{fig:closing_price_house_ares_color}
  \end{subfigure}
     \caption{Plots of house prices}
     \label{fig:house_plots}
\end{figure}


\newpage

\section*{Discussion}

In Table~\ref{tabular:five_number_summary}, the distribution of house closing prices seems to follow a Gaussian shape: the data is well distributed around 4,000,000 kr.
There is a small proportion of closing prices above 10,000,000 kr.

Figure~\ref{fig:closing_price_house_ares} shows, unsurprisingly, that increasing the house area increases the closing price on average.
We can also see that closing prices fluctuate more for larger areas than for smaller ones.

Finally, increasing the number of rooms tends to increase prices on average, which seems logical given that the floor area of a house is often linked to the number of rooms.

\newpage


\printbibliography

\section*{Appendix: Source Code}

\lstset{
  language=Python,
  basicstyle=\ttfamily,
  commentstyle=\color{OliveGreen},
  keywordstyle=\bfseries\color{Magenta},
  stringstyle=\color{YellowOrange},
  numbers=left,
  basicstyle=\footnotesize,
  breaklines=true,
  postbreak=\mbox{\textcolor{red}{$\hookrightarrow$}\space}
}


\lstinputlisting{ola/assignment2.py}

\end{document}
